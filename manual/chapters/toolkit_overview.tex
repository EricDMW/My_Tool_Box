\chapter{Toolkit Overview}

\section{Introduction to the Toolkit}

The \texttt{toolkit} package is a unified Python package that combines neural network components, plotting utilities, and parameter management tools into a single, cohesive research toolkit. It's designed to provide researchers and developers with a comprehensive set of tools for reinforcement learning and machine learning experiments.

\section{Package Structure}

The toolkit is organized into three main subpackages:

\begin{figure}[H]
\centering
\begin{verbatim}
toolkit/
├── __init__.py              # Main package initialization
├── neural_toolkit/          # Neural network components
│   ├── networks/            # Network architectures
│   ├── encoders/            # State encoders
│   ├── decoders/            # Output decoders
│   ├── discrete_tools/      # Tabular methods
│   └── utils/               # Utility functions
├── plotkit/                 # Plotting utilities
│   ├── core.py              # Core plotting functions
│   └── __main__.py          # Command-line interface
└── parakit/                 # Parameter management
\end{verbatim}
\caption{Toolkit package structure}
\end{figure}

\section{Neural Toolkit}

The \texttt{neural\_toolkit} subpackage provides comprehensive neural network components for reinforcement learning.

\subsection{Network Architectures}

\subsubsection{Policy Networks}

Policy networks implement different policy architectures:

\begin{itemize}
    \item \textbf{MLPPolicyNetwork}: Multi-layer perceptron policy
    \item \textbf{CNPolicyNetwork}: Convolutional neural network policy
    \item \textbf{RNNPolicyNetwork}: Recurrent neural network policy
    \item \textbf{TransformerPolicyNetwork}: Transformer-based policy
    \item \textbf{HybridPolicyNetwork}: Combination of different architectures
\end{itemize}

\subsubsection{Value Networks}

Value networks for state-value and action-value estimation:

\begin{itemize}
    \item \textbf{MLPValueNetwork}: MLP-based value network
    \item \textbf{CNValueNetwork}: CNN-based value network
    \item \textbf{RNNValueNetwork}: RNN-based value network
    \item \textbf{TransformerValueNetwork}: Transformer-based value network
\end{itemize}

\subsubsection{Q-Networks}

Q-networks for action-value function approximation:

\begin{itemize}
    \item \textbf{MLPQNetwork}: MLP-based Q-network
    \item \textbf{CNQNetwork}: CNN-based Q-network
    \item \textbf{RNNQNetwork}: RNN-based Q-network
    \item \textbf{TransformerQNetwork}: Transformer-based Q-network
    \item \textbf{DuelingQNetwork}: Dueling architecture Q-network
\end{itemize}

\subsection{State Encoders}

State encoders transform raw observations into neural network inputs:

\begin{itemize}
    \item \textbf{MLPEncoder}: Multi-layer perceptron encoder
    \item \textbf{CNNEncoder}: Convolutional neural network encoder
    \item \textbf{RNNEncoder}: Recurrent neural network encoder
    \item \textbf{TransformerEncoder}: Transformer-based encoder
    \item \textbf{HybridEncoder}: Combination of different encoders
\end{itemize}

\subsection{Output Decoders}

Output decoders transform network outputs into actions or values:

\begin{itemize}
    \item \textbf{MLPDecoder}: Multi-layer perceptron decoder
    \item \textbf{CNNDecoder}: Convolutional neural network decoder
    \item \textbf{RNNDecoder}: Recurrent neural network decoder
    \item \textbf{TransformerDecoder}: Transformer-based decoder
    \item \textbf{VariationalDecoder}: Variational autoencoder decoder
\end{itemize}

\subsection{Discrete Tools}

Tools for tabular reinforcement learning methods:

\begin{itemize}
    \item \textbf{QTable}: Q-learning table implementation
    \item \textbf{ValueTable}: Value iteration table
    \item \textbf{PolicyTable}: Policy table for discrete actions
    \item \textbf{DiscreteTools}: Utility functions for discrete RL
\end{itemize}

\section{Plotkit}

The \texttt{plotkit} subpackage provides advanced plotting utilities for research visualization.

\subsection{Core Features}

\begin{itemize}
    \item \textbf{Training Plots}: Loss curves, reward plots, and convergence analysis
    \item \textbf{Performance Metrics}: Accuracy, precision, recall, and F1-score plots
    \item \textbf{Network Analysis}: Weight distributions, activation maps, and gradients
    \item \textbf{Environment Visualization}: State representations and action distributions
    \item \textbf{Comparison Plots}: Multi-algorithm and multi-environment comparisons
\end{itemize}

\subsection{Plot Types}

\subsubsection{Training Visualization}

\begin{lstlisting}[language=python, caption=Training plot example]
from toolkit.plotkit import plot_training_curves

# Plot training progress
plot_training_curves(
    rewards=rewards_history,
    losses=loss_history,
    title="Training Progress",
    save_path="training_curves.png"
)
\end{lstlisting}

\subsubsection{Performance Analysis}

\begin{lstlisting}[language=python, caption=Performance analysis example]
from toolkit.plotkit import plot_performance_metrics

# Plot performance metrics
plot_performance_metrics(
    metrics={
        'accuracy': accuracy_history,
        'precision': precision_history,
        'recall': recall_history
    },
    title="Performance Metrics",
    save_path="performance.png"
)
\end{lstlisting}

\subsubsection{Network Analysis}

\begin{lstlisting}[language=python, caption=Network analysis example]
from toolkit.plotkit import plot_weight_distributions

# Plot weight distributions
plot_weight_distributions(
    model=policy_network,
    title="Weight Distributions",
    save_path="weights.png"
)
\end{lstlisting}

\section{Parakit}

The \texttt{parakit} subpackage provides parameter management and configuration tools.

\subsection{Features}

\begin{itemize}
    \item \textbf{Parameter Configuration}: JSON and YAML configuration files
    \item \textbf{Hyperparameter Management}: Grid search and random search utilities
    \item \textbf{Experiment Tracking}: Parameter logging and experiment management
    \item \textbf{Configuration Validation}: Parameter validation and type checking
\end{itemize}

\section{Usage Examples}

\subsection{Basic Usage}

\begin{lstlisting}[language=python, caption=Basic toolkit usage]
import torch
from toolkit.neural_toolkit import MLPPolicyNetwork, MLPValueNetwork
from toolkit.plotkit import plot_training_curves

# Create networks
policy_network = MLPPolicyNetwork(
    input_dim=10,
    output_dim=4,
    hidden_dims=[256, 256],
    activation='relu'
)

value_network = MLPValueNetwork(
    input_dim=10,
    hidden_dims=[256, 256],
    activation='relu'
)

# Training loop
rewards = []
losses = []

for episode in range(1000):
    # Training code here
    episode_reward = train_episode(policy_network, value_network)
    rewards.append(episode_reward)
    
    if episode % 100 == 0:
        loss = compute_loss(policy_network, value_network)
        losses.append(loss)

# Plot results
plot_training_curves(
    rewards=rewards,
    losses=losses,
    title="Training Progress",
    save_path="training_results.png"
)
\end{lstlisting}

\subsection{Advanced Usage}

\begin{lstlisting}[language=python, caption=Advanced toolkit usage]
import torch
from toolkit.neural_toolkit import (
    TransformerPolicyNetwork,
    TransformerValueNetwork,
    CNNEncoder,
    MLPDecoder
)
from toolkit.plotkit import plot_network_analysis

# Create complex network architecture
encoder = CNNEncoder(
    input_channels=3,
    hidden_dims=[32, 64, 128],
    output_dim=256
)

policy_network = TransformerPolicyNetwork(
    input_dim=256,
    output_dim=10,
    d_model=256,
    nhead=8,
    num_layers=6
)

value_network = TransformerValueNetwork(
    input_dim=256,
    d_model=256,
    nhead=8,
    num_layers=6
)

# Network analysis
plot_network_analysis(
    networks={
        'encoder': encoder,
        'policy': policy_network,
        'value': value_network
    },
    title="Network Architecture Analysis",
    save_path="network_analysis.png"
)
\end{lstlisting}

\section{Configuration}

\subsection{Default Configuration}

The toolkit uses sensible defaults for most parameters:

\begin{lstlisting}[language=python, caption=Default configuration]
DEFAULT_CONFIG = {
    'device': 'cuda' if torch.cuda.is_available() else 'cpu',
    'dtype': torch.float32,
    'seed': 42,
    'activation': 'relu',
    'dropout': 0.1,
    'layer_norm': False,
    'hidden_dims': [256, 256],
    'learning_rate': 0.001,
    'batch_size': 32
}
\end{lstlisting}

\subsection{Custom Configuration}

You can override default settings:

\begin{lstlisting}[language=python, caption=Custom configuration]
from toolkit import set_config

# Set custom configuration
set_config({
    'device': 'cuda:0',
    'dtype': torch.float16,
    'activation': 'gelu',
    'hidden_dims': [512, 512, 256],
    'learning_rate': 0.0001
})
\end{lstlisting}

\section{Best Practices}

\subsection{Network Design}

\begin{enumerate}
    \item Start with simple architectures and gradually increase complexity
    \item Use appropriate activation functions for your task
    \item Consider using layer normalization for training stability
    \item Monitor gradient flow and weight distributions
    \item Use appropriate learning rates and optimizers
\end{enumerate}

\subsection{Training}

\begin{enumerate}
    \item Use appropriate batch sizes for your hardware
    \item Monitor training progress with plots
    \item Implement early stopping to prevent overfitting
    \item Use learning rate scheduling for better convergence
    \item Save model checkpoints regularly
\end{enumerate}

\subsection{Visualization}

\begin{enumerate}
    \item Plot training curves to monitor progress
    \item Analyze network weights and activations
    \item Compare different architectures and hyperparameters
    \item Use appropriate plot types for different metrics
    \item Save plots with descriptive filenames
\end{enumerate}

\section{Integration with Environments}

The toolkit is designed to work seamlessly with the environment library:

\begin{lstlisting}[language=python, caption=Environment integration]
import gym
from toolkit.neural_toolkit import MLPPolicyNetwork
from env_lib import pistonball_env

# Create environment
env = pistonball_env.PistonballEnv()

# Create policy network
policy_network = MLPPolicyNetwork(
    input_dim=env.observation_space.shape[0],
    output_dim=env.action_space.n,
    hidden_dims=[256, 256]
)

# Training loop
for episode in range(1000):
    obs = env.reset()
    done = False
    
    while not done:
        action = policy_network.select_action(obs)
        obs, reward, done, info = env.step(action)
\end{lstlisting} 