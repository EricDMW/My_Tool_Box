\chapter{Installation and Setup}

\section{System Requirements}

\subsection{Operating System}
The toolkit supports the following operating systems:
\begin{itemize}
    \item Linux (Ubuntu 18.04+, CentOS 7+)
    \item macOS (10.14+)
    \item Windows (10+)
\end{itemize}

\subsection{Python Requirements}
\begin{itemize}
    \item Python 3.7 or higher
    \item pip package manager
    \item virtual environment (recommended)
\end{itemize}

\subsection{Hardware Requirements}
\begin{itemize}
    \item \textbf{Minimum}: 4GB RAM, 2GB free disk space
    \item \textbf{Recommended}: 8GB+ RAM, 5GB+ free disk space
    \item \textbf{GPU}: Optional but recommended for neural network training
\end{itemize}

\section{Environment Setup}

\subsection{Creating a Virtual Environment}

It's recommended to use a virtual environment to avoid dependency conflicts:

\begin{lstlisting}[language=bash, caption=Creating virtual environment]
# Create virtual environment
python -m venv my_toolbox_env

# Activate virtual environment
# On Linux/macOS:
source my_toolbox_env/bin/activate

# On Windows:
my_toolbox_env\Scripts\activate
\end{lstlisting}

\subsection{Installing Dependencies}

\subsubsection{Core Dependencies}

The toolkit requires several core Python packages:

\begin{lstlisting}[language=bash, caption=Installing core dependencies]
pip install torch>=1.9.0
pip install tensorflow>=2.6.0
pip install numpy>=1.20.0
pip install matplotlib>=3.5.0
pip install seaborn>=0.11.0
pip install scikit-learn>=1.0.0
pip install pandas>=1.3.0
pip install gym>=0.21.0
\end{lstlisting}

\subsubsection{Optional Dependencies}

For enhanced functionality, install these optional packages:

\begin{lstlisting}[language=bash, caption=Installing optional dependencies]
# For advanced plotting
pip install plotly>=5.0.0
pip install bokeh>=2.4.0

# For Jupyter notebook support
pip install jupyter>=1.0.0
pip install ipywidgets>=7.6.0

# For experiment tracking
pip install wandb>=0.12.0
pip install tensorboard>=2.8.0

# For parallel processing
pip install joblib>=1.1.0
pip install multiprocessing-logging>=0.3.0
\end{lstlisting}

\section{Installing the Toolkit}

\subsection{Installing Toolkit Project}

Navigate to the toolkit project directory and install in development mode:

\begin{lstlisting}[language=bash, caption=Installing toolkit project]
cd toolkit_project
pip install -e .
\end{lstlisting}

This installs the unified \texttt{toolkit} package containing:
\begin{itemize}
    \item \texttt{neural\_toolkit}: Neural network components
    \item \texttt{plotkit}: Plotting utilities
    \item \texttt{parakit}: Parameter management
\end{itemize}

\subsection{Installing Environment Library}

Install the environment library components:

\begin{lstlisting}[language=bash, caption=Installing environment library]
cd envlib_project
pip install -e .
\end{lstlisting}

\subsection{Installing Individual Environments}

You can also install environments individually:

\begin{lstlisting}[language=bash, caption=Installing individual environments]
# Pistonball environment
cd envlib_project/env_lib/pistonball_env
pip install -e .

# Kuramoto oscillator environment
cd envlib_project/env_lib/kos_env
pip install -e .

# Wireless communication environment
cd envlib_project/env_lib/wireless_comm_env
pip install -e .

# Agent-based lattice environment
cd envlib_project/env_lib/ajlatt_env
pip install -e .

# Linear message passing environment
cd envlib_project/env_lib/linemsg_env
pip install -e .
\end{lstlisting}

\section{Verification}

\subsection{Testing the Installation}

Create a simple test script to verify the installation:

\begin{lstlisting}[language=python, caption=Test installation script]
#!/usr/bin/env python3

def test_toolkit_imports():
    """Test toolkit imports"""
    try:
        import toolkit
        print("✓ Toolkit imported successfully")
        
        from toolkit import neural_toolkit
        print("✓ Neural toolkit imported successfully")
        
        from toolkit import plotkit
        print("✓ Plotkit imported successfully")
        
        return True
    except ImportError as e:
        print(f"✗ Import error: {e}")
        return False

def test_envlib_imports():
    """Test environment library imports"""
    try:
        import env_lib
        print("✓ Environment library imported successfully")
        
        # Test individual environments
        from env_lib import pistonball_env
        print("✓ Pistonball environment imported successfully")
        
        from env_lib import kos_env
        print("✓ Kuramoto environment imported successfully")
        
        return True
    except ImportError as e:
        print(f"✗ Import error: {e}")
        return False

if __name__ == "__main__":
    print("Testing My Tool Box installation...")
    print()
    
    toolkit_ok = test_toolkit_imports()
    print()
    envlib_ok = test_envlib_imports()
    
    if toolkit_ok and envlib_ok:
        print("\n🎉 All components installed successfully!")
    else:
        print("\n❌ Some components failed to install. Check the errors above.")
\end{lstlisting}

\subsection{Running the Test}

\begin{lstlisting}[language=bash, caption=Running installation test]
python test_installation.py
\end{lstlisting}

\section{Configuration}

\subsection{Environment Variables}

Set these environment variables for optimal performance:

\begin{lstlisting}[language=bash, caption=Setting environment variables]
# For PyTorch
export CUDA_VISIBLE_DEVICES=0  # Use first GPU
export OMP_NUM_THREADS=4       # Number of OpenMP threads

# For TensorFlow
export TF_CPP_MIN_LOG_LEVEL=2  # Reduce TensorFlow logging

# For matplotlib (if using headless server)
export MPLBACKEND=Agg
\end{lstlisting}

\subsection{Configuration Files}

Create configuration files for custom settings:

\begin{lstlisting}[language=python, caption=config.py example]
# Default configuration
DEFAULT_CONFIG = {
    'neural_toolkit': {
        'device': 'cuda' if torch.cuda.is_available() else 'cpu',
        'default_dtype': torch.float32,
        'seed': 42
    },
    'plotkit': {
        'style': 'seaborn-v0_8',
        'figure_size': (10, 6),
        'dpi': 100
    },
    'environments': {
        'render_mode': 'rgb_array',
        'max_episode_steps': 1000
    }
}
\end{lstlisting}

\section{Troubleshooting Installation}

\subsection{Common Issues}

\subsubsection{PyTorch Installation Issues}

If you encounter PyTorch installation problems:

\begin{lstlisting}[language=bash, caption=Fixing PyTorch installation]
# Remove existing PyTorch
pip uninstall torch torchvision torchaudio

# Install PyTorch with CUDA support (if available)
pip install torch torchvision torchaudio --index-url https://download.pytorch.org/whl/cu118

# Or install CPU-only version
pip install torch torchvision torchaudio --index-url https://download.pytorch.org/whl/cpu
\end{lstlisting}

\subsubsection{TensorFlow Installation Issues}

For TensorFlow problems:

\begin{lstlisting}[language=bash, caption=Fixing TensorFlow installation]
# Install TensorFlow with specific version
pip install tensorflow==2.10.0

# For GPU support
pip install tensorflow-gpu==2.10.0
\end{lstlisting}

\subsubsection{Permission Issues}

If you encounter permission errors:

\begin{lstlisting}[language=bash, caption=Fixing permission issues]
# Use user installation
pip install --user -e toolkit_project/

# Or use sudo (not recommended)
sudo pip install -e toolkit_project/
\end{lstlisting}

\subsection{Getting Help}

If installation issues persist:

\begin{enumerate}
    \item Check the system requirements
    \item Verify Python version compatibility
    \item Ensure all dependencies are installed
    \item Check for conflicting packages
    \item Review the troubleshooting chapter
\end{enumerate} 